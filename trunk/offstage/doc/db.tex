\section{Core Tables}

The core database works by keeping track of two basic things:

\begin{enumerate}
\item Entities (i.e. people or organizations)
\item Groups (arbitrary collections of entities).
\end{enumerate}

\subsection{Entities}

Entities in Offstage are the subjects on which we opreate --- either \emph{persons} or \emph{organizations} (stored in tables of those names).  Both subclass from the \emph{entities} table, sharing common fields in \emph{entities}.  Fields unique to each kind of object are stored in the subclassed tables.  Thus, the primary key field for ALL THREE tables is \emph{entityid}, and it is auto-generated.

\subsubsection{Families}

People occur in families.  In general, communication (eg mailings) takes place with the head of a family only, and only one mailing is sent per family (no matter how many kids are enrolled in the school).

The head of each family is denoted by the \emph{entities.primaryentityid} field in each family member.  For the head of the family, \emph{entityid} == \emph{primaryentityid}.

Also, \emph{entities.relprimarytypeid} describes the relationship with the family head --- see table \emph{relprimarytypes}.

\subsection{Groups}

A group is an arbitrary set of entities.  This is stored, most generally in the \emph{groups} table.  Assuming each group may be identified by a \emph{groupid} primary key, then this table associates (\emph{groupid}, \emph{entityid}) pairs.

The \emph{groupids} table gives a description for each group, in the form of a \emph{name} field.  This way, you can remember what group is what!

\subsubsection{Groups or Attributes}

Groups may be used in two ways:

\begin{enumerate}
 \item As a group of entities --- for example, the set of people involved in a mailing.
 \item As attributes attached to entities.  For example, a \emph{Nutcracker} group could be created, and all entities interested in \emph{The Nutcracker} be placed in it.  When displaying an entity, an on-screen field could show whether that person is in the \emph{Nutcracker} group.
 \end{enumerate}

\subsubsection{Group Subclasses}

Usually, groups need more information associated with them than just a name --- also, additional information may need to be associated with each entity in a group.  For example, a group called \emph{Donations 2005} could be used to track donors, but for each entity in the \emph{Donations 2005} group, one would need to keep track of how much money is donated.

This is accomplished through subclassing. Each \emph{type} of group \emph{xyz} --- needing its own kind of extra data --- is represented by two tables:
	\begin{enumerate}
 \item \emph{xyzids} subclasses from \emph{groupids}.  It describes each group of type \emph{xyz}.
 \item \emph{xyzs} is the actual grouping associations, and it subclasses from \emph{groups}.  Any extra fields to be associated with each entity go here.
 \end{enumerate}

The group types currently in the database are:
 \begin{enumerate}
\item[donation] Year-by-year donation data.  This should be replaced with a running donation table that gets filled in when each entity donates.
 \item[dtgroup] (?)
 \item[mailing] A mailing is a set of entities constructed at a particular point in time through queries.  Mailing labels may be printed.
 \item[note] Arbitrary annotation attached to entities. (\emph{notetypes} is obsolete)
 \item[phone] Phone numbers attached to entities
 \item[status] (?)
 \end{enumerate}

\section{Accounting \& Billing}

\subsection{Invoices}

  invoiceids

\subsection{Payments}
  paymentids

  cashpayments
  ccpayments
  checkpayments

\subsection{Allocating Payments to Invoices}
  paymentallocs

\section{School Management}

\subsection{Course Setup}

  termtypes
  terms
  courseids
  meetings

\subsection{Registrations and Programs}

  regelig
  registrations

  programids
  coursesetids
  coursesets

\subsection{Enrollment --- YDP and Open Class}

  enrollments
  courseroletypes
  subs





Queries
=======

Queries are represented by Java data structures, which are stored in an XML serialization format.  They are not stored in the database.

PGA Tables
==========

All tables starting in \emph{pga_} are created by the \emph{pgaccess} front-end program.  They may safely be deleted.


Import Tables
=============
Import tables all start with an i: idonationdata, iinforequests, etc.  They are used only to import data from MS Access, and are not really part of the DB schema.
